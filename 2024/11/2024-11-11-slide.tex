\documentclass{beamer}
\usepackage{ctex} % 支持中文
\usepackage{graphicx}
\usepackage{amsmath}
\usepackage{booktabs}
\usepackage{hyperref}

% 添加参考文献支持
\bibliographystyle{alpha}

\usetheme{Madrid}
\usecolortheme{seahorse}

% 自定义块颜色
\setbeamercolor{block title}{bg=blue!30,fg=black}
\setbeamercolor{block body}{bg=blue!10,fg=black}
\setbeamercolor{exampleblock title}{bg=green!50,fg=black}
\setbeamercolor{exampleblock body}{bg=green!20,fg=black}

% 开启图表编号
\setbeamertemplate{caption}[numbered]

% 标题、作者、日期
\title{\textbf{周报-向嘉豪(2024-11-11)}}
\author{向嘉豪}
\institute{衡阳师范学院}
\date{2024年11月11日}

\begin{document}

\begin{frame}
    \titlepage
\end{frame}

\begin{frame}
    \frametitle{摘要}
    \begin{block}{本周工作}
        \begin{itemize}
            \item 完成了线性层的重构
            \item 系统性分析了\cite{Leurent2024}的线性层优化方法
            \item 实验结果表明,循环矩阵在AES中的应用效果未达预期
            \item 基于此,我们将研究重点转向线性层中置换操作的优化
        \end{itemize}
    \end{block}
\end{frame}

\section{线性层优化算法分析}
\begin{frame}
    \frametitle{线性层优化算法分析}
    \begin{block}{基本概念}
        \begin{itemize}
            \item 初始状态:$((x_1), 1)$
            \item 代价函数:$Cost(x) = weight(x)$,即$x$的汉明权重
            \item 通过状态转移递归地优化代价函数
        \end{itemize}
    \end{block}
    \begin{block}{基本转移规则}
        \begin{enumerate}
            \item $x_{i} = 1 \lll r : ((x_1, \dots, x_{i}, \dots, x_v), v) \rightarrow ((x_1, \dots, x_{i-1}, x_{i+1}, \dots, x_v), v-1)$
            \item $x_{i} = x_{j} \lll r : ((x_1, \dots, x_{i}, \dots, x_v), v) \rightarrow ((x_1, \dots, x_{i-1}, x_{i+1}, \dots, x_v), v-1)$
        \end{enumerate}
    \end{block}
\end{frame}

\begin{frame}
    \frametitle{核心转移策略}
    \begin{block}{三种核心策略}
        \begin{enumerate}
            \item $x_i = a \oplus (a \ggg r) \oplus b$
            \begin{itemize}
                \item $a = x_i \land (x_i \lll r)$
                \item $a \land (a \ggg r) = 0$
            \end{itemize}
            \item $x_{i} = x_{i} \oplus (x_{j} \lll r)$,$i \neq j$
            \item $x_i = a \oplus b$,$x_j = (a \ggg r) \oplus c$
        \end{enumerate}
    \end{block}
\end{frame}

\section{AES线性层优化实现}
\begin{frame}
    \frametitle{AES线性层结构分析}
    \begin{block}{矩阵结构}
        \begin{itemize}
            \item 基于\cite{Adomnicai2021}的研究,分析了切片AES线性层$L=MP$
            \item $M$为$128 \times 128$矩阵,具有显著的分块特征
            \item $P$为$128 \times 128$单位置换矩阵
        \end{itemize}
    \end{block}
    \begin{exampleblock}{矩阵表示}
        \small{
        $M = \begin{pmatrix}
            M_0 & 0 & 0 & 0 \\
            0 & M_0 & 0 & 0 \\
            0 & 0 & M_0 & 0 \\
            0 & 0 & 0 & M_0
        \end{pmatrix}$,其中

        $M_0 = \begin{pmatrix}
            M_{00} & M_{01} & M_{02} & M_{03} \\
            M_{03} & M_{00} & M_{01} & M_{02} \\
            M_{02} & M_{03} & M_{00} & M_{01} \\
            M_{01} & M_{02} & M_{03} & M_{00}
        \end{pmatrix}$
        }
    \end{exampleblock}
\end{frame}

\section{优化效果分析}
\begin{frame}
    \frametitle{优化效果分析}
    \begin{block}{实验验证}
        \begin{itemize}
            \item 对AES第一个寄存器的$M$矩阵在交错形式下进行分析
            \item 可表示为$4\times8$矩阵$M_i$
        \end{itemize}
    \end{block}
    \begin{exampleblock}{矩阵$M_i$}
        \small{
        $M_i = \begin{pmatrix}
            0 & 1 & 1 & 1 & 1 & 0 & 1 & 0 \\
            1 & 0 & 0 & 1 & 1 & 1 & 1 & 0 \\
            1 & 0 & 1 & 0 & 0 & 1 & 1 & 1 \\
            1 & 1 & 1 & 0 & 1 & 0 & 0 & 1
        \end{pmatrix}$
        }
    \end{exampleblock}
\end{frame}

\begin{frame}
    \frametitle{实验结果}
    \begin{block}{发现}
        \begin{itemize}
            \item 向量$x_i = 01111010$可实现最优分解:$x_i = a \oplus (a \ggg r) \oplus b$
            \item 参数:$a = 0101000$,$r = 3$,$b = 0010000$
            \item 在交错形式下,$Cost(a) = 2$,可以减少1次XOR
            \item 然而,转换回标准形式后,仍需4次XOR操作
        \end{itemize}
    \end{block}
    \begin{block}{结论}
        揭示了\cite{Leurent2024}优化方法的局限性,仅在XOR操作次数超过4次时体现优势
    \end{block}
\end{frame}

\section{论文撰写}
\begin{frame}
    \frametitle{论文撰写}
    \begin{block}{研究结论}
        \begin{itemize}
            \item 循环矩阵在AES应用中效果未达预期
            \item 将研究重心转向线性层中置换操作的优化
        \end{itemize}
    \end{block}
    \begin{block}{主要工作}
        \begin{enumerate}
            \item \textbf{结构优化}:
            \begin{itemize}
                \item 重新组织内容结构
                \item 提高论文的逻辑性和可读性
            \end{itemize}
            \item \textbf{算法改进}:
            \begin{itemize}
                \item 基于切片并行和动态规划的策略,优化merge和split操作
            \end{itemize}
            \item \textbf{OPO算法优化}:
            \begin{itemize}
                \item 引入贪心递归策略
                \item 确保算法收敛性和全局最优解
            \end{itemize}
        \end{enumerate}
    \end{block}
\end{frame}

\begin{frame}[allowframebreaks]
    \frametitle{参考文献}
    \bibliography{../../paper}
\end{frame}

% last is the adviser comment and our answer, one comment one answer and let them in one block

\begin{frame}
    \frametitle{老师评语}
    \begin{block}{一、工作报告应明确呈现结果和工作量}
        本周主要精力投入在理解\cite{Leurent2024}的算法,并尝试将其应用于AES的线性层优化。由于效果不理想,报告中未详细阐述该部分工作
    \end{block}
    \begin{block}{二、需明确最终完成计划}
        本周计划,完成AES实现、实验数据的整理和论文初稿
    \end{block}
\end{frame}

\end{document}