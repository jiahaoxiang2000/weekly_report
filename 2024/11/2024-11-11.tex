\documentclass[11pt,a4paper]{article}
\usepackage{ctex}
\usepackage[utf8]{inputenc}
\usepackage{geometry}
\usepackage{fancyhdr}
\usepackage{enumitem}
\usepackage{titlesec}
\usepackage{amsmath} % Added for math equation support
\usepackage{titling}
\usepackage[hidelinks]{hyperref}
% \usepackage[style=authoryear]{biblatex} % Use biber backend
% \addbibresource{../../paper.bib} % Specify the .bib file
% Define \lll
\newcommand{\lll}{\mathrel{<\!\!<\!\!<}}
% Define \ggg
\newcommand{\ggg}{\mathrel{>\!\!>\!\!>}}
\geometry{margin=0.5in}
\titleformat{\section}{\large\bfseries}{\thesection}{0.5em}{}

% title context and style setting
\title{周报-向嘉豪(2024-11-11)}
\setlength{\droptitle}{-6em} % Reduce space begin the title
% Redefine \maketitle to display only the title
\renewcommand{\maketitle}{
  \begin{center}
    \LARGE\bfseries\thetitle
  \end{center}
}

\begin{document}

\maketitle


% Custom Abstract
\noindent \textbf{Abstract}: 本周主要工作为线性层部分重写。


\section{线性层部分重写}

为了深入理解线性层的优化方法,我们研究了\cite{Leurent2024}。由于\cite{Leurent2024}中对线性层的描述较为简略,我们结合其提供的源代码进行了详细分析。

\subsection{学习\cite{Leurent2024}源码}

未优化的线性层最初表示为:$((x_1), 1)$,其代价函数为:$Cost(x) = weight(x)$,即$x$的汉明距离。随后,作者采用递归方法,根据不同条件进行状态转移,确保代价函数逐步收敛。终止条件如下:

\begin{align*}
x_{i} = 1 \lll r & : ((x_1, \dots, x_{i}, \dots, x_v), v) \rightarrow ((x_1, \dots, x_{i-1}, x_{i+1}, \dots, x_v), v-1) \\
x_{i} = x_{j} \lll r & : ((x_1, \dots, x_{i}, \dots, x_v), v) \rightarrow ((x_1, \dots, x_{i-1}, x_{i+1}, \dots, x_v), v-1)
\end{align*}

每次转移后,代价函数的取值降低。然而,作者未对选择这些转移条件的原因进行说明。我们认为,作者采用了启发式方法,无法保证优化结果为最优。具体的转移条件如下:

\begin{align*}
x_i &= a \oplus (a \ggg r) \oplus b, \quad a = x_i \land (x_i \lll r), \quad a \land (a \ggg r) = 0 : \nonumber \\
&\quad ((x_1, \dots, x_{i}, \dots, x_v), v) \rightarrow ((x_1, \dots, a, \dots, x_v, b), v+1) \text{ or } ((x_1, \dots, a, \dots, x_v), v) \\
x_{i} &= x_{i} \oplus (x_{j} \lll r), \quad i \neq j : ((x_1, \dots, x_{i}, \dots, x_v), v) \rightarrow ((x_1, \dots, x_{i} \oplus x_{j} \lll r, \dots, x_v), v) \\
x_i &= a \oplus b, \quad x_j = (a \ggg r) \oplus c : ((x_1, \dots, x_{i}, \dots, x_{j}, \dots, x_v), v) \rightarrow ((x_1, \dots, b, \dots, c, \dots, x_v, a), v+1)
\end{align*}

% Include the bibliography
\bibliographystyle{alpha}
\bibliography{../../paper}

\end{document}