\documentclass[11pt,a4paper]{article}
\usepackage{ctex}
\usepackage[utf8]{inputenc}
\usepackage{geometry}
\usepackage{fancyhdr}
\usepackage{enumitem}
\usepackage{titlesec}
\usepackage{amsmath} % Added for math equation support
\usepackage{titling}
\usepackage[hidelinks]{hyperref}
% \usepackage[style=authoryear]{biblatex} % Use biber backend
% \addbibresource{../../paper.bib} % Specify the .bib file
% Define \lll
\newcommand{\lll}{\mathrel{<\!\!<\!\!<}}
% Define \ggg
\newcommand{\ggg}{\mathrel{>\!\!>\!\!>}}
\geometry{margin=0.5in}
\titleformat{\section}{\large\bfseries}{\thesection}{0.5em}{}

% title context and style setting
\title{周报-向嘉豪(2024-11-11)}
\setlength{\droptitle}{-6em} % Reduce space begin the title
% Redefine \maketitle to display only the title
\renewcommand{\maketitle}{
  \begin{center}
    \LARGE\bfseries\thetitle
  \end{center}
}

\begin{document}

\maketitle


% Custom Abstract
\noindent \textbf{Abstract}: 本周完成了线性层的重构。为深入理解线性层的优化方法,我们对\cite{Leurent2024}进行了系统性分析。实验结果表明,循环矩阵在AES中的应用效果未达预期。基于此发现,我们将研究重点转向线性层中置换操作的优化,主要包括以下三个方面:结构优化、算法改进和OPO算法优化。

\noindent \textbf{下周计划}: 1)完善AES算法实现的实验工作。

\subsection{线性层优化算法分析}

线性层的初始状态表示为$((x_1), 1)$,其中代价函数定义为$Cost(x) = weight(x)$,表示输入向量$x$的汉明权重。优化过程采用递归方法,通过状态转移实现代价函数的单调递减。基本转移规则包含以下两类:

\begin{align}
x_{i} = 1 \lll r & : ((x_1, \dots, x_{i}, \dots, x_v), v) \rightarrow ((x_1, \dots, x_{i-1}, x_{i+1}, \dots, x_v), v-1) \\
x_{i} = x_{j} \lll r & : ((x_1, \dots, x_{i}, \dots, x_v), v) \rightarrow ((x_1, \dots, x_{i-1}, x_{i+1}, \dots, x_v), v-1)
\end{align}

进一步分析表明,算法采用了三种核心转移策略:

\begin{align}
x_i &= a \oplus (a \ggg r) \oplus b, \quad a = x_i \land (x_i \lll r), \quad a \land (a \ggg r) = 0 : \nonumber \\
&\quad ((x_1, \dots, x_{i}, \dots, x_v), v) \rightarrow ((x_1, \dots, a, \dots, x_v, b), v+1) \text{ 或 } ((x_1, \dots, a, \dots, x_v), v) \\
x_{i} &= x_{i} \oplus (x_{j} \lll r), \quad i \neq j : ((x_1, \dots, x_{i}, \dots, x_v), v) \rightarrow ((x_1, \dots, x_{i} \oplus x_{j} \lll r, \dots, x_v), v) \\
x_i &= a \oplus b, \quad x_j = (a \ggg r) \oplus c : ((x_1, \dots, x_{i}, \dots, x_{j}, \dots, x_v), v) \rightarrow ((x_1, \dots, b, \dots, c, \dots, x_v, a), v+1)
\end{align}

\subsection{AES线性层优化实现}

基于\cite{Adomnicai2021}的研究,我们分析了切片AES线性层$L=MP$的结构特征。其中$M$为$128 \times 128$矩阵,$P$为$128 \times 128$单位置换矩阵。$M$具有显著的分块特征:

\begin{align}
M &= \begin{pmatrix}
    M_0 & 0 & 0 & 0 \\
    0 & M_0 & 0 & 0 \\
    0 & 0 & M_0 & 0 \\
    0 & 0 & 0 & M_0
\end{pmatrix}, \quad \text{其中} \quad
M_0 = \begin{pmatrix}
    M_{00} & M_{01} & M_{02} & M_{03} \\
    M_{03} & M_{00} & M_{01} & M_{02} \\
    M_{02} & M_{03} & M_{00} & M_{01} \\
    M_{01} & M_{02} & M_{03} & M_{00}
\end{pmatrix}
\end{align}

\subsection{优化效果分析}

对AES第一个寄存器的$M$矩阵在interleaved形式下进行分析,其可表示为$4\times8$矩阵$M_i$:

\begin{align}
M_i = \begin{pmatrix}
    0 & 1 & 1 & 1 & 1 & 0 & 1 & 0 \\
    1 & 0 & 0 & 1 & 1 & 1 & 1 & 0 \\
    1 & 0 & 1 & 0 & 0 & 1 & 1 & 1 \\
    1 & 1 & 1 & 0 & 1 & 0 & 0 & 1
\end{pmatrix}
\end{align}

通过实验验证,我们发现:

\begin{itemize}
\item 向量$x_i = 01111010$可实现最优分解:$x_i = a \oplus (a \ggg r) \oplus b$
\item 参数取值:$a = 0101000$,$r = 3$,$b = 0010000$
\item interleaved形式下,$Cost(a) = 2$,理论上可减少33\%的XOR操作
\end{itemize}

然而,实验结果表明,当转换回标准形式时,仍需4次XOR操作。这一现象揭示了\cite{Leurent2024}优化方法的局限性:仅在XOR操作次数超过4次时才能体现实质性优势

\subsection{论文撰写}

本周研究表明,循环矩阵在AES应用中的效果未达预期。基于这一发现,我们将研究重心转向线性层中置换操作的优化,主要包含以下三个方面:

\textbf{结构优化:}为提高论文的逻辑性和可读性,我们重新组织了内容结构:首先引入置换操作的基本概念和理论基础,其次详细阐述置换操作的优化方法,最后展示在实际应用中的优化效果。

\textbf{算法改进:} 在split和merge操作的设计中,我们基于两个核心思想进行优化:切片并行,通过数据分割提高计算效率,引出merge;动态规划,采用自底向上的优化策略,引出split。

\textbf{OPO算法优化:} 我们对原有的OPO(Optimal Permutation Operation)算法进行了改进:引入贪心递归策略,保证算法收敛性,优化分解与合并过程,确保获得全局最优解。

% Include the bibliography
\bibliographystyle{alpha}
\bibliography{../../paper}

\end{document}