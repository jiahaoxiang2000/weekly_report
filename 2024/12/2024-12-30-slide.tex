% filepath: /Users/xiangjiahao/tex/weekly_report/2024/12/2024-12-30-slide.tex
\documentclass{beamer}
\usepackage{ctex}
\usepackage[utf8]{inputenc}
\usepackage{graphicx}
\usepackage{amsmath}
\usepackage{amssymb}
\usepackage{booktabs}
\usepackage{hyperref}
\usepackage{subcaption}
\usepackage{multicol}
\usepackage{listings}
\usepackage{multirow}
\usepackage{tabularx}

\bibliographystyle{alpha}

\usetheme{Madrid}
\usecolortheme{seahorse}

% 自定义块颜色
\setbeamercolor{block title}{bg=blue!30,fg=black}
\setbeamercolor{block body}{bg=blue!10,fg=black}
\setbeamercolor{alertblock title}{bg=red!50,fg=black}
\setbeamercolor{alertblock body}{bg=red!20,fg=black}

% 开启图表编号
\setbeamertemplate{caption}[numbered]

\title{\textbf{周报——向嘉豪(2024年12月31日)}}
\author{向嘉豪}
\institute{衡阳师范学院}
\date{2024年12月31日}

\begin{document}

\begin{frame}
    \titlepage
\end{frame}

\begin{frame}
    \frametitle{摘要}
    \begin{block}{}
        \begin{enumerate}
            \item 复现\cite{Lee2022}的实验
            \item 完善论文写作,初步确定题目
        \end{enumerate}

    \end{block}
\end{frame}



\section{论文实验}

\begin{frame}
    \frametitle{实验复现与结果}
    \begin{block}{在RTX 4090平台上复现\cite{Lee2022}实验}
        作者通过数据表示方式的重排降低了线程间等待,并对部分轮密钥的预计算,使AES的并行效率提升。使用其实现对1GB消息加密,吞吐量可达3057 Gbps,较\cite{Lee2022}中RTX 3080平台(1489 Gbps)提升近一倍,说明了硬件平台对整体性能的关键影响。表~\ref{tab:aes_gpu_compare}呈现了当下的实验结果。
    \end{block}
    \begin{table}[h]
        \centering
        \caption{基于GPU的AES CTR模式实现性能对比}
        \label{tab:aes_gpu_compare}
        \begin{tabular}{cccc}
        \toprule
        \textbf{实现} & \textbf{吞吐量 (Gbps)} & \textbf{硬件平台} & \textbf{发表年份} \\
        \midrule
        \cite{Hajihassani2019} & 1,478 & Tesla V100 & 2019 \\
        \cite{Lee2022}        & 1,489 & RTX 3080   & 2022 \\
        本文复现            & 3,057 & RTX 4090   & --- \\
        \bottomrule
        \end{tabular}
   
        \end{table}
\end{frame}

\section{论文写作}

\begin{frame}
    \frametitle{论文写作进展}
    \begin{block}{论文题目《High Throughput Implementation of AES on GPUs》,(其中AES算法需要替换), 并完成了引言与摘要部分的写作}
        引言突出对称加密在大规模数据传输场景(如数据中心与5G网络)实现高吞吐的紧迫需求,同时结合GPU并行计算的潜力,对现有研究的局限性与挑战进行概述。后续将在论文中重点介绍bitslicing与线程调度的优化方法。
    \end{block}
\end{frame}

\begin{frame}[allowframebreaks]
    \frametitle{参考文献}
    \bibliography{../../paper}
\end{frame}

\begin{frame}
    \frametitle{老师评语}

    \begin{alertblock}{不要用AES,建议用最新顶刊或顶会的一个密码算法(最好是别人没做过的)}
        寻找一个新的密码算法
    \end{alertblock}
    \begin{block}{本周计划}
        \begin{enumerate}
            \item 继续深入阅读\cite{Lee2022}的GPU实现代码
            \item 寻找最新顶会或顶刊的密码算法
        \end{enumerate} 
    \end{block}
\end{frame}

\end{document}