% filepath: /Users/xiangjiahao/tex/weekly_report/2024/12/2024-12-30.tex
\documentclass[11pt,a4paper]{article}
\usepackage{ctex}
\usepackage[utf8]{inputenc}
\usepackage{geometry}
\usepackage{fancyhdr}
\usepackage{enumitem}
\usepackage{titlesec}
\usepackage{amsmath} % Added for math equation support
\usepackage{amssymb}
\usepackage{graphicx} % Added for including graphics
\usepackage{titling}
\usepackage{subcaption}
\usepackage{multicol}
\usepackage{listings}
\usepackage{booktabs}
\usepackage{multirow}
\usepackage[hidelinks]{hyperref}

\geometry{margin=0.5in}
\titleformat{\section}{\large\bfseries}{\thesection}{0.5em}{}
\title{周报-向嘉豪(2024-12-30)}
\setlength{\droptitle}{-6em}

\renewcommand{\maketitle}{
  \begin{center}
    \LARGE\bfseries\thetitle
  \end{center}
}

\begin{document}

\maketitle

\noindent \textbf{摘要}:围绕GPU加速下AES实现的实验与论文写作开展工作。通过阅读\cite{Lee2022}并复现其在RTX 4090环境下的实验设置,我们成功获得了对1GB消息加密可达3057 Gbps的速度。同时,我们也初步确定了论文的题目《High Throughput Implementation of AES on GPUs》,并对引言部分进行了初步写作。

\vspace{1em}
\noindent \textbf{下周计划}:1)仔细阅读\cite{Lee2022}的代码,和\cite{Hajihassani2019}论文,2)继续论文写作,思考我们优化实现的方案。

\section{论文实验}

本周主要对\cite{Lee2022}一文进行了细致的阅览与实验复现。作者在该文中通过对数据表示方式的重新排列,去除了线程间运行的等待时间,并预计算部分轮密钥,从而有效提升AES的并行效率。我们依照其公开的实验配置在RTX 4090平台上进行测试时,针对1GB消息加密达到了3057 Gbps,相较\cite{Lee2022}原文在RTX 3080中测得的1489 Gbps提升近一倍。我们推测,硬件平台的性能差异在其中发挥了主导性作用。表~\ref{tab:aes_gpu_compare}展示了现阶段的实验结果。我们之后的目标是将吞吐量提升至3057 Gbps以上。

\begin{table}[htbp]
    \caption{基于GPU的AES CTR模式实现性能对比}
    \label{tab:aes_gpu_compare}
    \centering
    \begin{tabular}{cccc}
        \toprule
        \textbf{实现} & \textbf{吞吐量 (Gbps)} & \textbf{硬件平台} & \textbf{发表年份} \\
        \midrule
        \cite{Hajihassani2019} & 1,478 & Tesla V100 & 2019 \\
        \cite{Lee2022}        & 1,489 & RTX 3080   & 2022 \\
        本文复现    & 3,057 & RTX 4090   & --- \\
        \bottomrule
    \end{tabular}
\end{table}

\section{论文写作}
我们初步确定了题目《High Throughput Implementation of AES on GPUs》,并对引言与摘要写作。引言部分强调了AES在大规模数据传输(如数据中心、5G网络等)环境下实现高吞吐的紧迫需求,结合GPU大规模并行计算的潜能性,对当前研究的差距和挑战进行初步分析。为解决传统软件与扩展指令架构在Gbps到Tbps量级数据吞吐范围内的局限性,我们计划在论文中集中展示bitslicing技术与线程调度方法的优化思路,以缓解缓存未命中和线程阻塞对性能的影响。

\bibliographystyle{alpha}
\bibliography{../../paper}
\end{document}