\documentclass{beamer}
\usepackage{ctex}
\usepackage[utf8]{inputenc}
\usepackage{graphicx}
\usepackage{amsmath}
\usepackage{amssymb}
\usepackage{booktabs}
\usepackage{hyperref}
\usepackage{subcaption}
\usepackage{multicol}
\usepackage{listings}
\usepackage{multirow}
\usepackage{tabularx}

\bibliographystyle{alpha}

\usetheme{Madrid}
\usecolortheme{seahorse}
\setbeamertemplate{caption}[numbered]
% 自定义块颜色
\setbeamercolor{block title}{bg=blue!30,fg=black}
\setbeamercolor{block body}{bg=blue!10,fg=black}
\setbeamercolor{alertblock title}{bg=red!50,fg=black}
\setbeamercolor{alertblock body}{bg=red!20,fg=black}

\title{\textbf{周报}}
\author{向嘉豪}
\institute{衡阳师范学院}
\date{\today}

\begin{document}

\begin{frame}
  \titlepage
\end{frame}

\begin{frame}
  \frametitle{摘要}
  \begin{itemize}
    \item \textbf{论文阅读}:实现算法更换为SPHINCS\textsuperscript{+}
    \item \textbf{论文写作}:确定题目, 完成引言部分写作
  \end{itemize}
\end{frame}

\begin{frame}
  \frametitle{论文阅读}
  \noindent \textbf{更换实现算法: } 鉴于后量子密码标准化进程的重要进展,我们决定将研究重心转向NIST后量子密码标准化项目。该项目于2024年8月13日公布了最终标准,包括CRYSTALS-Dilithium、CRYSTALS-KYBER和SPHINCS\textsuperscript{+}等算法。在学长的指导下,我们选择了SPHINCS\textsuperscript{+}作为研究对象,这是一个\textcolor{blue}{无状态哈希签名方案},由\cite{Bernstein2019}提出。与传统数字签名不同,SPHINCS\textsuperscript{+}基于哈希函数构建,能够\textcolor{red}{抵抗量子计算攻击},在后量子密码标准化中具有重要地位。我们计划基于其第三轮提交规范开展优化实现工作。

\end{frame}

\begin{frame}
  \frametitle{论文阅读}

  \noindent \textbf{SPHINCS\textsuperscript{+}算法: } SPHINCS\textsuperscript{+}的签名生成过程包括三个主要步骤:计算消息哈希值,FORS签名和HT签名。其中表~\ref{tab:hash_latency}展示了SPHINCS\textsuperscript{+}中哈希函数的延迟测试结果。其中\textcolor{blue}{H、F和Hmsg}分别表示HT签名、FORS签名和Hmsg的哈希函数延迟。\textcolor{blue}{PRF、PRFmsg} 为计算过程中伪随机数生成所需延迟。为此我们可以从HT、FORS、Hmsg和PRF四个方面考虑,以求更优的实现方案。

  \begin{table}[htbp]
    \centering
    \caption{SPHINCS\textsuperscript{+}-128F-SIMPLE 哈希函数延迟测试(微秒)\cite{Wang2025}}
    \label{tab:hash_latency}
    \begin{tabular}{lccccccc}
      \toprule
      算法 & H & F & PRF & PRFmsg & Hmsg \\
      \midrule
      SHA-256  & 3.2 & 2.8 & 1.6 & 5.9 & 4.8 \\
      SHAKE256  & 6.9 & 6.5 & 5.1 & 5.2 & 6.3 \\
      \bottomrule
    \end{tabular}
  \end{table}
\end{frame}

\begin{frame}
  \frametitle{论文写作}
  \begin{itemize}
    \item 我们确定了\textcolor{blue}{题目}《Efficient Implementations of post-quantum SPHINCS\textsuperscript{+} on GPUs》, (i.e, 老师指导下,添加后量子算法) ,并完成了摘要部分的撰写。
    \item \textcolor{blue}{引言}部分阐述了量子计算对现有密码体系的威胁,强调了后量子密码学标准化进程中SPHINCS\textsuperscript{+}作为无状态哈希签名方案的重要地位。结合SPHINCS\textsuperscript{+}计算开销大的特点,我们提出利用GPU并行计算能力来加速签名生成过程。
  \end{itemize}
\end{frame}

\begin{frame}[allowframebreaks]
  \frametitle{参考文献}
  \bibliography{../../paper}
\end{frame}
\begin{frame}
  \frametitle{老师评语}
  \begin{alertblock}{题目中能不能在这个算法修改为 后量子算法sphincs}
    已添加后量子算法
  \end{alertblock}
  \begin{block}{下周计划}
    \begin{enumerate}
      \item 研读SPHINCS\textsuperscript{+}第三轮提交规范及实现代码,整理关键数据结构和操作流程.
      \item 深入分析GPU端并行化策略\cite{Wang2025}.
    \end{enumerate}
  \end{block}

\end{frame}
\end{document}