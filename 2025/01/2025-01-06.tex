\documentclass[11pt,a4paper]{article}
\usepackage{ctex}
\usepackage[utf8]{inputenc}
\usepackage{geometry}
\usepackage{fancyhdr}
\usepackage{enumitem}
\usepackage{titlesec}
\usepackage{amsmath} % Added for math equation support
\usepackage{amssymb}
\usepackage{graphicx} % Added for including graphics
\usepackage{titling}
\usepackage{subcaption}
\usepackage{multicol}
\usepackage{listings}
\usepackage{booktabs}
\usepackage{multirow}
\usepackage[hidelinks]{hyperref}

\geometry{margin=0.5in}
\titleformat{\section}{\large\bfseries}{\thesection}{0.5em}{}
\title{周报-向嘉豪(2025-01-06)}
\setlength{\droptitle}{-6em}

\renewcommand{\maketitle}{
  \begin{center}
    \LARGE\bfseries\thetitle
  \end{center}
}

\begin{document}

\maketitle

\noindent \textbf{摘要}:
本周完成了三项主要工作:1)确定了研究方向,选择NIST后量子密码标准化项目中的SPHINCS\textsuperscript{+}算法作为实现对象;2)完成了初步技术调研,分析了SPHINCS\textsuperscript{+}中哈希函数的性能瓶颈,测试显示SHA-256和SHAKE256在不同操作中的延迟范围在1.6-31.0微秒之间;3)开始论文写作,确定了题目《Efficient Implementations of SPHINCS\textsuperscript{+} on GPUs》,并完成了引言部分初稿。通过分析发现,SPHINCS\textsuperscript{+}的签名生成过程具有明显的并行计算特性,这为我们利用GPU进行优化实现提供了可能。

\vspace{1em}
\noindent \textbf{下周计划}: 1)研读SPHINCS\textsuperscript{+}第三轮提交规范及参考实现代码,整理关键数据结构和操作流程. 2) 深入分析GPU端并行化策略\cite{Wang2025}.

\section{论文阅读}
\noindent \textbf{更换实现算法: } 鉴于后量子密码标准化进程的重要进展,我们决定将研究重心转向NIST后量子密码标准化项目。该项目于2024年8月13日公布了最终标准,包括CRYSTALS-Dilithium、CRYSTALS-KYBER和SPHINCS\textsuperscript{+}等算法。在学长的指导下,我们选择了SPHINCS\textsuperscript{+}作为研究对象,这是一个无状态哈希签名方案,由\cite{Bernstein2019}提出。与传统数字签名不同,SPHINCS\textsuperscript{+}基于哈希函数构建,能够抵抗量子计算攻击,在后量子密码标准化中具有重要地位。我们计划基于其第三轮提交规范开展优化实现工作。

\noindent \textbf{SPHINCS\textsuperscript{+}算法: } SPHINCS\textsuperscript{+}的签名生成过程包括三个主要步骤:计算消息哈希值,FORS签名和HT签名。其中表~\ref{tab:hash_latency}展示了SPHINCS\textsuperscript{+}中哈希函数的延迟测试结果。其中H、F和Hmsg分别表示HT签名、FORS签名和Hmsg的哈希函数延迟。PRF、PRFmsg 为计算过程中随机数生成所需延迟。为此我们可以从HT、FORS、Hmsg和PRF四个方面考虑,以求更优的实现方案。

\begin{table}[htbp]
\centering
\caption{SPHINCS+-128F-SIMPLE 哈希函数延迟测试(微秒)\cite{Wang2025}}
\label{tab:hash_latency}
\begin{tabular}{lccccccc}
\toprule
算法 & H & F & PRF & PRFmsg & Hmsg \\
\midrule
SHA-256  & 3.2 & 2.8 & 1.6 & 5.9 & 4.8 \\
SHAKE256  & 6.9 & 6.5 & 5.1 & 5.2 & 6.3 \\
\bottomrule
\end{tabular}
\end{table}

\noindent

\section{论文写作}
\begin{itemize}
\item 我们确定了题目《Efficient Implementations of SPHINCS\textsuperscript{+} on GPUs》, 并完成了摘要部分的撰写。
\item 引言部分阐述了量子计算对现有密码体系的威胁,强调了后量子密码学标准化进程中SPHINCS\textsuperscript{+}作为无状态哈希签名方案的重要地位。结合SPHINCS\textsuperscript{+}计算开销大的特点,我们提出利用GPU并行计算能力来加速签名生成过程。
\end{itemize}

\bibliographystyle{alpha}
\bibliography{../../paper}
\end{document}