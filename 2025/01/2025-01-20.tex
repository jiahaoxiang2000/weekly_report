\documentclass[report]{../../custom}
\begin{document}
\maketitle

\section{Coding reading}

On the last weekly, we had to read the paper and do some coding, we had figured out the SPHINCS+ all the compent, we will use the SPX to repalce the SPHINCS+, for the SPX, it have the three compent.

on the top level see, the SPX have the arithy length mssage $msg$ for it inupt, then on the signer have it own sercurity key $sk_{seed}$, and public key $pk_{seed}$. So the auth message, it like $SPX_{sign}: (msg,sk_{seed}) \mapsto (pk_{root},auth)$. then we will by the detial to saw the signature proposs.

first the $msg$ to the hash function, i.e. which function can chose differen sercurity level, by the hash fucntion it will out put the one have value $hm$, $tree_{index}$, for the FORS sign and HT sign (muti XMSS tree) respectily.

then we have the $n$ bytes lenght of the $hm$, this is the hash function output the fix value, for the specific verison SPX. So we will do the FORS singture,
first spilt the $8\times n$ bit $hm$ dividel by the FORS tree height $t$, where the tree leaf node number is $2^t$. here $t | (8\times n)$. the $k\times t=(8\times n)$, the $k$ is the number of FORS tree. on the FORS leaf node is the random creat the $sk$ by the $sk_{seed}$ and other field.
all the leaf node by the $sk_{1\dots k}$ by the hash function for the low height level node, then the $k$ tree will split to compute the different root node. finally use the all the root node hash to the $FORS_{pk}$. on the FORS the $auth$, is the aside the node by the $hm$ spilt to $k$ index.
we use the figure to show the FORS auth node and pulic node.

% here use tkiz to draw the tree. here have three 4 node 2 height binary tree. then all the root to link to one father node.

\begin{tikzpicture}[level distance=1.5cm,
    level 1/.style={sibling distance=3.5cm},
    level 2/.style={sibling distance=2.5cm},
  every node/.style={circle, draw, fill=white, inner sep=2pt}]

  % Root node
  \node (root) {}
  % First level
  child {node  {}
    % Second level
    child {node  {}}
    child {node  {}}
  }
  child {node  {}
    % Second level
    child {node {}}
    child {node {}}
  }
  child {node  {}
    % Second level
    child {node [fill=red!100]  {}}
    child {node [fill=blue!100]  {} child {node [fill=green!100] {}}}
  };

\end{tikzpicture}

\end{document}
