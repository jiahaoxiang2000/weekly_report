\documentclass[slide]{../../custom}
\usepackage{makecell}
\begin{document}

\begin{frame}
  \titlepage
\end{frame}

\begin{frame}
  \frametitle{摘要}
  \begin{itemize}
    \item \textbf{SHA256 GPU实现与性能分析}
    \item \textbf{论文前置知识写作}
  \end{itemize}
\end{frame}

\begin{frame}
  \frametitle{SHA256 实验结果}
  \begin{table}
    \centering
    \caption{不同实现方式的最大吞吐量对比}
    \begin{tabular}{|l|c|c|c|}
      \hline
      实现方式 & \makecell{最大吞吐量\\(MB/s)} & \makecell{消息大小\\(B)} & 加速比 \\
      \hline
      CPU单核 \cite{Wang2025} & 230.40 & 131,072 & 1× \\
      GPU单核 \cite{Wang2025} & 25.39 & 16,384 & 0.11× \\
      \hline
      GPU并行 & 478,324.05 & 1,024 & 2,076× \\
      \hline
      GPU多流 & 22,923.71 & 16,384 & 99× \\
      \hline
    \end{tabular}
  \end{table}

  \begin{itemize}
    \item GPU单核性能低于CPU单核:执行效率低+数据传输开销
    \item 最优配置(Grid=128, Block=256):\textcolor{blue}{2076倍加速比}
  \end{itemize}
\end{frame}

\begin{frame}
  \frametitle{性能分析}
  \begin{columns}
    \column{0.48\textwidth}
    \textbf{线程配置影响}
    \begin{itemize}
      \item Grid=128匹配SM数量最优
      \item Block=256在资源平衡点
      \item 大Block导致资源竞争
    \end{itemize}

    \column{0.48\textwidth}
    \textbf{消息大小影响}
    \begin{itemize}
      \item 4B-512B:快速增长期
      \item 1024B-4096B:峰值性能
      \item >4096B:性能下降期
    \end{itemize}
  \end{columns}

  \vspace{0.5cm}
  \textcolor{blue}{关键发现:动态调整线程配置和消息分配具有研究价值}, 即存在核函数与其并发数量的最优组合,该组合配置下,吞吐量最大。 \\
\end{frame}

\begin{frame}
  \frametitle{前置知识写作}
  \textbf{SPHINCS\textsuperscript{+}组成}
  \begin{itemize}
    \item WOTS+:一次性签名方案
    \item FORS:少次签名方案
    \item Hypertree:多层Merkle树结构
  \end{itemize}

  \vspace{0.3cm}
  \textbf{GPU计算模型}
  \begin{itemize}
    \item 硬件:多个SM,每个包含CUDA核心
    \item 多级存储系统
    \item CUDA优化策略
  \end{itemize}
\end{frame}

\begin{frame}
  \frametitle{老师评语}
  \begin{alertblock}{继续推进}
  \end{alertblock}
  \begin{block}{下周计划}
  1) 将动态线程配置策略扩展到整个SPHINCS\textsuperscript{+}签名过程 \\
2) 探索GPU多流处理技术,进一步提高性能
\end{block}
\end{frame}

\begin{frame}
\frametitle{参考文献}
\bibliographystyle{alpha}
\bibliography{../../paper}
\end{frame}

\end{document}