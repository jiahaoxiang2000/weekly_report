\documentclass[slide]{../../custom}
\usepackage{makecell}
\begin{document}

\begin{frame}
  \titlepage
\end{frame}

\begin{frame}
  \frametitle{摘要}
  \begin{itemize}
    \item \textbf{SHA256哈希算法的GPU并行实现与优化}
    \item \textbf{论文研究动机、贡献和实现部分的修订}
  \end{itemize}
\end{frame}

\begin{frame}
  \frametitle{SHA256线程束级并行实现}
  \begin{columns}
    \column{0.48\textwidth}
    \textbf{线程布局设计}
    \begin{itemize}
      \item 线程0:状态初始化
      \item 线程0-15:消息字加载与调度扩展
      \item 线程0-7:轮计算中的状态变量管理
      \item 线程0:填充和最终输出
    \end{itemize}

    \column{0.48\textwidth}
    \textbf{优化亮点}
    \begin{itemize}
      \item 充分利用线程束内并行性
      \item 减少线程束分化
      \item \texttt{\_\_shfl\_sync()}实现高效数据共享
      \item 吞吐量约120MB/s,\textcolor{blue}{6倍提升}, \cite{Wang2025}
    \end{itemize}
  \end{columns}
\end{frame}

\begin{frame}
  \frametitle{论文研究动机更新}
  \textbf{强化研究必要性}
  \begin{itemize}
    \item \textbf{现有实现的局限}
      \begin{itemize}
        \item 注重最大化吞吐量,忽视单线程执行效率
        \item Kim等人:多次内核启动导致效率低下
        \item Wang等人的CUSPX:线程利用和资源管理有优化空间
      \end{itemize}
    \item \textbf{关键观察}
      \begin{itemize}
        \item 底层哈希函数优化不足
        \item 未充分考虑线程数量与执行效率间的权衡
      \end{itemize}
  \end{itemize}
\end{frame}

\begin{frame}
  \frametitle{论文主要贡献修订}
  \begin{enumerate}
    \item \textbf{哈希函数级并行方法}
      \begin{itemize}
        \item 细粒度任务分配减少延迟
        \item 显著加速SPHINCS+核心计算原语
      \end{itemize}

    \item \textbf{自适应线程分配策略}
      \begin{itemize}
        \item 优化线程数量与内核函数效率间平衡
        \item 最小化同步开销
        \item 最大化GPU计算吞吐量
      \end{itemize}
  \end{enumerate}
\end{frame}

\begin{frame}
  \frametitle{老师评语}
  \begin{alertblock}{工作量偏小,现在的写作我通篇看了下,离快报差距非常大}
    \begin{itemize}
      \item 计划增加工作深度:完善SHA256实现细节,补充性能分析与对比实验
      \item 提高写作质量:重构论文框架,完善技术细节,强化创新点阐述
    \end{itemize}
  \end{alertblock}
  \begin{block}{下周计划}
    \begin{enumerate}
      \item 基于优化后的SHA256实现,完成SPHINCS+签名方案中WOTS+和FORS组件的GPU加速实现
      \item 设计并实现自适应线程分配策略,针对不同参数集优化计算资源分配
    \end{enumerate}
  \end{block}
\end{frame}

\begin{frame}
  \frametitle{参考文献}
  \bibliographystyle{alpha}
  \bibliography{../../paper}
\end{frame}

\end{document}
