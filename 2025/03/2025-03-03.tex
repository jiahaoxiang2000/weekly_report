\documentclass[report]{../../custom}
\begin{document}
\maketitle

\noindent \textbf{摘要:}本周主要工作为SHA256哈希算法的GPU并行实现与优化。通过线程束级别的细粒度并行方法,实现了哈希函数内部状态的高效并行处理。并进行了系统性能评估。测试结果显示,优化后的SHA256实现在数据处理时可达160MB/s的吞吐量,相比现有实现提升约6倍。此外,完成了论文中研究动机、主要贡献和哈希函数级并行化实现部分的修订与完善。

\vskip 0.5cm

\noindent \textbf{下周计划:} 1) 基于优化后的SHA256实现,完成SPHINCS+签名方案中WOTS+和FORS组件的GPU加速实现,着重对比密钥生成阶段;2) 设计并实现自适应线程分配策略,针对不同参数集优化计算资源分配。

\section{SHA256并行实现}

本周主要工作为在GPU上实现SHA256哈希算法的高效并行计算。SHA256作为SPHINCS+签名方案的基础组件,其性能优化对整体签名速度至关重要。

\subsection{线程束级并行实现}

通过线程束(warp)级并行实现单个哈希计算,
线程布局设计如下:线程0负责状态初始化工作;线程0-15协同并行加载消息字;线程0-15共同处理消息调度扩展;线程0-7分别管理轮计算中的不同状态变量;最后由线程0处理填充和最终输出操作。通过这种精细的任务分配,我们充分利用了线程束内的并行性,同时避免了不必要的线程间通信开销。

主要优势在于:更好地利用单个哈希操作的资源,通过同步操作减少线程束分化,并使用\texttt{\_\_shfl\_sync()}等线程束级原语实现高效的数据共享。测试表明在处理大量消息时能达到约120MB/s的吞吐量,相比\cite{Wang2025}有\textcolor{blue}{6倍的提升}。

\section{论文写作}

\subsection{研究动机更新}

对论文第1.2节中的研究动机进行了更新和完善,主要从以下两个角度强化了我们工作的必要性:
首先,指出现有实现主要关注于通过广泛并行化来最大化吞吐量,但往往忽视了单个线程执行的效率。Kim等人的实现证明了并行处理的潜力,但由于多次内核启动而效率低下,而Wang等人的CUSPX虽然引入了全面的并行框架,但其线程利用和资源管理仍有优化空间。

其次,提出了驱动本研究的两个关键观察:(1)现有实现通常集中于并行化SPHINCS+算法结构,而对构成该方案计算核心的底层哈希函数优化关注不足;(2)现有实现往往优先考虑最大线程并行性,而没有充分考虑线程数量和执行效率之间的权衡,导致由于同步开销增加、内存访问延迟和单线程计算效率降低而性能不佳。

\subsection{主要贡献修订}

修订了论文第1.3节中的前两点贡献,使其更加明确和具体:
第一点贡献强调了我们提出的哈希函数级并行方法,通过细粒度任务分配减少延迟,显著加速了SPHINCS+的核心计算原语。这一贡献直接针对当前SPHINCS+实现中哈希函数处理效率低下的问题。

第二点贡献阐述了我们开发的自适应线程分配策略,优化了线程数量和内核函数效率之间的平衡,在GPU架构上最小化同步开销的同时最大化计算吞吐量。这一贡献解决了现有实现中对线程资源使用不当导致的性能问题。

\subsection{哈希函数级并行化实现}

编写并完善了第3.1节"哈希函数级并行化"的详细内容,这是本文的核心技术创新之一。该部分主要包括:我们的哈希函数级并行化主要从三个方面进行优化。首先是状态初始化优化,多个线程同时初始化哈希函数状态数组的不同部分,减少初始化开销。具体来说,线程0处理初始状态设置,线程0-15协作并行加载消息字。其次是轮函数优化,SHA256置换的每一轮被分解为可由不同线程并发执行的通道操作。线程0-15处理消息调度扩展,线程0-7管理轮计算中的状态变量更新。最后是数据共享优化,通过warp级原语(如\texttt{\_\_shfl\_sync()})实现高效数据共享,无需昂贵的共享内存操作。同步操作确保warp内的线程可以在没有warp分化的情况下交换数据,提高计算效率。

\bibliographystyle{alpha}
\bibliography{../../paper}

\end{document}