\documentclass[slide]{../../custom}
\usepackage{makecell}
% Emoji configuration (common for both modes)

\begin{document}

\begin{frame}
  \titlepage
\end{frame}

\begin{frame}
  \frametitle{摘要}
  \begin{itemize}
    \item \textbf{XMSS树结构的并行实现优化}
    \item \textbf{论文的写作}
  \end{itemize}
\end{frame}

\begin{frame}
  \frametitle{XMSS树并行实现优化}
  XMSS树是HT签名的基本组件,是签名中最耗时的部分。
  % auto set the vertical space
  \vfill
  \begin{columns}
    \column{0.48\textwidth}
    \textbf{两层次并行技术}
    \begin{itemize}
      \item 节点级并行
        \begin{itemize}
          \item 同一层次的节点并行计算
          \item \textcolor{blue}{二级并行}实现 \cite{Wang2025}
        \end{itemize}
      \item WOTS级并行
        \begin{itemize}
          \item 单个节点内的WOTS链并行计算
          \item \textcolor{blue}{三级并行}实现 \cite{Wang2025}
        \end{itemize}
    \end{itemize}

    \column{0.48\textwidth}
    \textbf{动态调度与GPU优化}
    \begin{itemize}
      \item 运行期间动态调整线程数
      \item 结合GPU特性进行性能优化
    \end{itemize}
  \end{columns}
\end{frame}

\begin{frame}
  \frametitle{实验结果}
  \centering
  \begin{tabular}{|l|c|c|}
    \hline
    \textbf{操作} & \textbf{执行时间} & \textbf{性能提升} \\
    \hline
    PKGEN 串行实现 \cite{Wang2025} & 31.382 ms & 基准 \\
    \hline
    PKGEN 二级并行 \cite{Wang2025} & 3.844 ms & 8.16x \\
    \hline
    PKGEN 二级并行+动态调度 & 3.822 ms & 8.21x \\
    \hline
    PKGEN 二三级并行 \cite{Wang2025} & 0.220 ms & 142.65x \\
    \hline
    PKGEN 二三级并行+动态调度+GPU优化 & 0.197 ms & \textcolor{blue}{159.30x} \\
    \hline
  \end{tabular}

  \vspace{0.5cm}
  \textbf{关键发现:}
  \begin{itemize}
    \item 二级并行中动态调度提升有限 (8.16x → 8.21x)
    \item 二三级并行中动态调度效果显著 (142.65x → 159.30x)
    \item 执行时间优化:0.220ms → 0.197ms (\textcolor{red}{10.9\%}提升)
  \end{itemize}
\end{frame}

\begin{frame}
  \frametitle{论文写作️}
  \begin{itemize}
    \item \textbf{核心理论基础:} \textcolor{blue}{动态调度}作为性能提升的理论依据
    \item \textbf{关键发现:} 不同核函数$g$存在最优线程数$t$使性能达到最大
    \item \textbf{数学模型:}
      \begin{itemize}
        \item 签名过程表示为运行序列$((g_1,t_1),\dots,(g_n,t_n))$
        \item 目标: 构建映射函数\textcolor{blue}{$F:G\rightarrow T$}
        \item 为每个核函数$g_i$分配最优线程配置$t_i$
      \end{itemize}
  \end{itemize}
  \vfill

  \begin{enumerate}
    \item \textbf{最优线程配置映射函数$F$}
      \begin{itemize}
        \item 基于核函数特性自动确定最优线程数
        \item 减少人工调优需求,提高通用性
      \end{itemize}

    \item \textbf{完整调度实现机制}
      \begin{itemize}
        \item 运行时线程分配与管理
        \item 针对不同签名组件的专用优化
        \item 最大化GPU资源利用率
      \end{itemize}
  \end{enumerate}
\end{frame}

\begin{frame}
  \frametitle{老师评语}

  \begin{alertblock}{继续加快推进,特别写作}
    反复打磨论文写作
  \end{alertblock}
  \vfill
  \begin{block}{下周计划}
    \begin{enumerate}
      \item 最优线程配置映射函数 $F$ 写作
      \item 拓展动态调度实现到FORS等组件
    \end{enumerate}
  \end{block}

\end{frame}

\begin{frame}
  \frametitle{参考文献}
  \bibliographystyle{alpha}
  \bibliography{../../paper}
\end{frame}

\end{document}
