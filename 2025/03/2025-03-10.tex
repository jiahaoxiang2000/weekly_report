\documentclass[report]{../../custom}
\begin{document}
\maketitle

\noindent \textbf{摘要:}
XMSS树结构的并行实现优化及相关论文的写作与重构上。针对SPHINCS+签名方案中的XMSS组件,我们实施了节点级并行和WOTS级并行两层次的并行技术,并结合动态调度策略进行优化。实验结果表明,在二级并行中加入动态调度带来了轻微的性能提升(8.16x至8.21x),而在三级并行中采用动态调度与GPU优化技术后,性能显著提高(142.65x至159.30x),PKGEN操作的执行时间从0.220ms降至0.197ms,实现了10.9\%的性能提升。同时,我们确定论文的理论基础,将动态调度作为性能提升的理论依据,去构建针对不同核函数的最优线程配置映射函数,并设计完整的调度实现机制。

\vskip 0.5cm

\noindent \textbf{下周计划:} 1) 构建最优线程配置映射函数,强化理论分析部分;2)对剩余组件进行动态调度的实现。

\section{XMSS树并行实现优化}

SPHINCS+签名方案中的XMSS(树哈希)组件是影响整体性能的关键因素。本周我们重点优化了XMSS树结构的并行实现,主要采用了二个层次的并行技术:\textbf{节点级并行(Node-based parallelism)}:同一层次的节点并行计算, 此为\cite{Wang2025}中实现的\textcolor{blue}{二级并行}。\textbf{WOTS级并行(WOTS-based parallelism)}:单个节点内的WOTS链并行计算, 此为\cite{Wang2025}中实现的\textcolor{blue}{三级并行}。

实验结果表明,这些并行优化技术显著提高了签名性能。以下是在参数配置为\texttt{n = 24, h = 66, d = 22, b = 8, k = 33, w = 16, len = 51}的SLH-DSA-SHA-256-192f算法上的基准测试结果:

\begin{table}[ht]
  \centering
  \begin{tabular}{|l|c|c|}
    \hline
    \textbf{操作} & \textbf{执行时间} & \textbf{性能提升} \\
    \hline
    PKGEN 串行实现 \cite{Wang2025} & 31.382 ms & 基准 \\
    \hline
    PKGEN 二级并行 \cite{Wang2025} & 3.844 ms & 8.16x \\
    \hline
    PKGEN 二级并行+动态调度 & 3.822 ms & 8.21x \\
    \hline
    PKGEN 二三级并行 \cite{Wang2025} & 0.220 ms & 142.65x \\
    \hline
    PKGEN 二三级并行+动态调度+GPU优化 & 0.197 ms & 159.30x \\
    \hline
  \end{tabular}
  \caption{XMSS树并行实现的性能对比}
  \label{tab:xmss_performance}
\end{table}

首先我们将\textcolor{blue}{动态调度}(在运行的过程中调整运行的线程数,而非固定线程数去并行运行),而将其放入到\textcolor{blue}{二级并行}中,性能只从8.16x提升到8.21x, 这说明动态调度在二级并行中并没有显著的性能提升。为此我们猜测是由于WOTS节点串型实现影响了性能提升。我们将\textcolor{blue}{动态调度}放入到\textcolor{blue}{三级并行}中,性能从142.65x提升到159.30x, 这说明动态调度在三级并行中有显著的性能提升。同时我们结合了GPU优化技术,对XMSS树生成签名进行优化,从0.220 ms提升到0.197 ms, 有\textcolor{red}{10.9\%}的性能提升。

\section{论文写作}

在论文重构过程中,我们发现现有的创新点缺乏理论基础。为此,我们决定将\textcolor{blue}{动态调度}作为性能提升的理论依据,同时补充实验数据以提高工作的重要性。我们之前的实验表明,在同一GPU平台上,对于不同核函数$g$,存在最优线程数$t$使性能达到最大。这一发现成为我们动态调度方法的理论基础。

在整个签名过程中,签名过程可表示为一个运行序列$((g_1,t_1),\dots,(g_n,t_n))$,我们的目标是使整体性能最优。由于签名算法是确定的,我们已知$(g_1,\dots,g_n)$,需要确定$(t_1,\dots,t_n)$。因此,我们的核心创新点是构建映射函数\textcolor{blue}{$F:G\rightarrow T$},它能为每个核函数$g_i$分配最优线程配置$t_i$。
为实现对签名各组件的动态调度,我们会设计一套完整的\textcolor{blue}{调度实现}机制。这一实现机制是我们工作的第二个创新点,为动态调度策略提供了实践基础。

\bibliographystyle{alpha}
\bibliography{../../paper}

\end{document}