\documentclass[slide]{../../custom}
\usepackage{makecell}
% Emoji configuration (common for both modes)

\begin{document}

\begin{frame}
  \titlepage
\end{frame}

\begin{frame}
  \frametitle{摘要}
  \begin{itemize}
    \item \textbf{自适应线程分配(Adaptive Thread)技术探索}
    \item \textbf{论文绪论和自适应线程部分的修改完善}
  \end{itemize}
\end{frame}

\begin{frame}
  \frametitle{自适应线程分配理论模型}
  \begin{columns}
    \column{0.48\textwidth}
    \textbf{传统问题}
    \begin{itemize}
      \item 通常将线程设为可用\textcolor{red}{最大值}
      \item 忽略\textcolor{blue}{函数特性}差异
      \item 资源分配\textcolor{red}{不均衡}
    \end{itemize}

    \textbf{理论模型}
    \begin{itemize}
      \item \textcolor{blue}{自适应线程函数}$AT:G\rightarrow T$
      \item 性能模型: $T(g_i, t) = \textcolor{blue}{\alpha_i + \frac{\beta_i}{t} + \gamma_i \cdot t}$
      \item 最优线程计算: $\textcolor{red}{t_i^* = \sqrt{\frac{\beta_i}{\gamma_i}}}$
    \end{itemize}

    \column{0.48\textwidth}
    \textbf{参数含义}
    \begin{itemize}
      \item $\alpha_i$: \textcolor{blue}{固定开销}
      \item $\frac{\beta_i}{t}$: \textcolor{blue}{并行加速部分}
      \item $\gamma_i \cdot t$: \textcolor{blue}{线程管理开销}
      \item $t$: \textcolor{blue}{线程数量}
    \end{itemize}

    \textbf{创新点}
    \begin{itemize}
      \item 根据函数特性\textcolor{red}{动态分配}
      \item \textcolor{blue}{平衡并行与管理开销}
      \item 资源\textcolor{red}{最优化利用}
    \end{itemize}
  \end{columns}
\end{frame}

\begin{frame}
  \frametitle{实验结果}
  \centering
  \begin{tabular}{ccc}
    \toprule
    \makecell{\textbf{Block×Thread}\\\textbf{配置}} & \makecell{\textbf{每操作}\\\textbf{耗时(ms)}} & \makecell{\textbf{相对默认配置}\\\textbf{性能提升}} \\
    \midrule
    128×64 & 0.0025 & -13.6\% \\
    默认配置 (512×32) \cite{Wang2025} & 0.0022 & 基准 \\
    128×256 & 0.0017 & \textcolor{blue}{22.7\%} \\
    128×512 & 0.0018 & 18.2\% \\
    \bottomrule
  \end{tabular}

  \vspace{0.5cm}
  \textbf{关键发现(公钥生成):}
  \begin{itemize}
    \item 最优线程配置(128×256)提升22.7\%性能
    \item 并非最大线程数(128×512)才有最优性能
    \item 验证了理论模型的有效性
  \end{itemize}
\end{frame}

% \begin{frame}
%   \frametitle{实施策略}
%   \begin{enumerate}
%     \item \textbf{离线性能剖析}
%       \begin{itemize}
%         \item 精确量化不同线程配置下的执行效率
%         \item 识别关键函数的最优线程配置
%       \end{itemize}

%     \item \textbf{优化映射表构建}
%       \begin{itemize}
%         \item 为常见操作预先确定理想配置
%         \item 建立函数特征到线程参数的映射关系
%       \end{itemize}

%     \item \textbf{动态线程分配机制}
%       \begin{itemize}
%         \item 运行时根据计算负载自动调整
%         \item 考虑硬件状态和并行度需求
%         \item 实现接近最优的性能表现
%       \end{itemize}
%   \end{enumerate}
% \end{frame}

\begin{frame}
  \frametitle{论文写作进展}
  学长指导下,我们查阅了NIST文献,确认SPHINCS+已于2024年8月更名为SLH-DSA并纳入FIPS 205标准。
  \vfill
  \begin{columns}
    \column{0.48\textwidth}
    \textbf{绪论部分修改}
    \begin{itemize}
      \item 精简\textcolor{blue}{PQC背景}介绍
      \item 突出\textcolor{red}{SPHINCS+与SLH-DSA关系}
      \item 重构\textcolor{blue}{相关工作}部分
      \item 明确当前GPU实现的效率瓶颈:
        \begin{itemize}
          \item \textcolor{red}{统一的最大线程分配策略}
          \item \textcolor{red}{次优的线程性能}
        \end{itemize}
      \item 强化\textcolor{blue}{贡献点}表述
    \end{itemize}

    \column{0.48\textwidth}
    \textbf{自适应线程分配部分}
    \begin{itemize}
      \item 构建基于\textcolor{blue}{函数特性}的性能模型
      \item 提出\textcolor{red}{最优线程数}计算公式
      \item 设计\textcolor{blue}{动态实现}算法
      \item \textcolor{red}{离线分析}与\textcolor{red}{运行时调整}相结合
      \item 资源\textcolor{blue}{最优利用}方案
    \end{itemize}
  \end{columns}
\end{frame}

\begin{frame}
  \frametitle{老师评语}
  \begin{alertblock}{加快}
  预计4月底,完成初稿。
  \end{alertblock}
  \vfill
  \begin{block}{下周计划}
    \begin{enumerate}
      \item 完善实验和创新点二写作
      \item 设计并实现自动化框架,用于确定各类密码算法操作的最佳线程配置
    \end{enumerate}
  \end{block}


\end{frame}

\begin{frame}
  \frametitle{参考文献}
  \bibliographystyle{alpha}
  \bibliography{../../paper}
\end{frame}

\end{document}
