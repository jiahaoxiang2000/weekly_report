\documentclass[report]{../../custom}
\begin{document}
\maketitle

\noindent \textbf{摘要:}本周对SPHINCS+签名算法进行了动态并行优化研究,重点探索了自适应线程分配(Adaptive Thread)技术。通过理论分析提出了线程性能模型$T(g_i, t) = \alpha_i + \frac{\beta_i}{t} + \gamma_i \cdot t$,并导出最优线程数计算公式$t_i^* = \sqrt{\frac{\beta_i}{\gamma_i}}$。实验验证表明,对于密钥生成操作,最优线程配置(128×256)相比默认配置(512×32)提升了22.7\%的性能。同时完成了论文的绪论和自适应线程分配部分的修改,优化了论文结构和学术论证。

\vskip 0.5cm

\noindent \textbf{下周计划:} 1)将自适应线程研究扩展到签名和验证操作;2) 设计并实现自动化框架,用于确定各类密码算法操作的最佳线程配置。

\section{实验验证}

\subsection{Adaptive Thread}

本周我们针对SPHINCS+签名算法进行了动态并行优化研究,重点探索了自适应线程分配(Adaptive Thread)技术在提升吞吐量方面的效果。

\subsubsection{理论分析}

传统的CUDA并行实现通常将线程数量设为可用的最大值,但我们的研究表明这并非最优方案。对于每个内核函数$g_i$,存在一个最佳的线程数量$t_i$,可以使其性能达到最大化。我们提出了自适应线程函数$AT:G\rightarrow T$,它将每个函数$g_i \in G$映射到其最优线程数$t_i \in T$。

为准确定义此映射,我们通过经验性能建模方法进行研究:

\begin{equation}
T(g_i, t) = \alpha_i + \frac{\beta_i}{t} + \gamma_i \cdot t
\end{equation}

其中:
$\alpha_i$ 表示固定开销,$\frac{\beta_i}{t}$ 表示并行加速部分,$\gamma_i \cdot t$ 表示线程管理开销,而 $t$ 是线程数量。

通过微分计算,可以得到函数$g_i$的最优线程数为:

\begin{equation}
t_i^* = \sqrt{\frac{\beta_i}{\gamma_i}}
\end{equation}

\subsubsection{实验结果}

我们采用CUSPX实现对SPHINCS+算法进行了不同线程配置的性能测试,重点关注密钥生成操作的执行效率。表\ref{tab:thread_comparison}展示了不同thread-block配置下的性能对比。

\begin{table}[htbp]
\centering
\caption{不同线程配置下SPHINCS+密钥生成性能对比}
\label{tab:thread_comparison}
\begin{tabular}{|c|c|c|}
  \hline
  \textbf{Block×Thread配置} & \textbf{每操作耗时(ms)} & \textbf{相对默认配置性能提升} \\
  \hline
  默认配置 (512×32) \cite{Wang2025} & 0.0022 & 基准 \\
  \hline
  128×256 & 0.0017 & 22.7\% \\
  \hline
  128×64 & 0.0025 & -13.6\% \\
  \hline
  128×512 & 0.0018 & 18.2\% \\
  \hline
\end{tabular}
\end{table}

实验结果表明,对于SPHINCS+的密钥生成操作,128×256的线程配置比默认的512×32配置提供了22.7\%的性能提升。这验证了我们的理论假设,即并非所有操作都应使用最大线程数。特别是,在RTX-4090这样的硬件上,尽管理论上有128个SM×1024线程可用,但并非所有计算类型都能从最大化线程数中获益。

\subsubsection{实施策略}

基于上述研究,我们提出了以下实施策略:首先对每个关键函数进行离线性能剖析,精确量化不同线程配置下的执行效率;然后构建函数到最优线程数的映射表,为常见操作预先确定理想配置;最后实现动态线程分配机制,在运行时根据具体计算负载、硬件状态和并行度需求自动调整线程参数,从而在各种操作场景下实现接近最优的性能表现。

接下来,我们将扩展这项研究到签名和验证操作,并开发一个自动化框架来确定各种密码操作的最佳线程配置。

\section{论文写作}

本周完成了论文的两个关键部分的修改和完善:

\begin{enumerate}
\item \textbf{绪论部分}:精简了PQC背景介绍,突出SPHINCS+与SLH-DSA的关系;重构了相关工作部分,明确指出当前GPU实现的两个主要效率瓶颈:统一的最大线程分配策略和次优的线程性能;强化了论文贡献点的表述,突出自适应线程分配方法的创新性。

\item \textbf{自适应线程分配部分}:构建了基于函数特性的性能建模方法,采用$T(g_i, t) = \alpha_i + \frac{\beta_i}{t} + \gamma_i \cdot t$模型来平衡并行加速与线程管理开销;提出最优线程数计算公式$t_i^* = \sqrt{\frac{\beta_i}{\gamma_i}}$;设计了动态实现算法,通过离线分析和运行时调整实现资源的最优利用。
\end{enumerate}

\bibliographystyle{alpha}
\bibliography{../../paper}

\end{document}