\documentclass{article}
\usepackage{../../typesetting/styles/report-zh}
\usepackage{threeparttable} % Add this package for tablenotes environment

\setCJKmainfont{Source Han Serif SC} % Main Chinese font (Songti)
\setCJKsansfont{Source Han Sans SC} % Sans-serif Chinese font
\setCJKmonofont{Noto Sans Mono CJK SC} % Monospaced Chinese font (uncomment if needed)

% Set document information
\title{周报~向嘉豪 (\today)}
\author{向嘉豪}
\date{\today}

\begin{document}

\maketitle

\begin{abstract}
  本周主要完成了\blue{论文的修订定稿}与\blue{IEEE期刊投稿准备工作}。论文方面,对Thread-Adaptive: Optimized Parallel Architectures of SLH-DSA on GPUs进行了系统性优化,明确了\red{线程自适应分配架构}与\red{函数级并行方法}的技术细节,完善了实验数据与性能分析。投稿准备方面,已按照\blue{IEEE TCAS-II}期刊要求完成了Cover Letter撰写、文档整理与格式规范化等工作。
\end{abstract}

\begin{weekplan}
1) \blue{论文校正与投稿}
\end{weekplan}

\section{论文修正和投稿准备}

\subsection{论文修正}

本周对论文Thread-Adaptive: Optimized Parallel Architectures of SLH-DSA on GPUs进行了最终修订与完善,主要内容包括:
对论文架构的核心贡献进行了更清晰的阐述。将三层并行优化架构的描述进行了精炼,包括\blue{自适应线程分配(ATA)}与\blue{函数级并行(FLP)}两个关键技术点。自适应线程分配技术通过建立精确的性能模型,为每个密码操作精确校准最佳线程配置,有效平衡了并行计算与同步开销。函数级并行方法将密码操作分解为细粒度计算任务,实现了更高效的资源利用。这两项技术的结合显著提升了SLH-DSA在GPU平台上的执行性能。

完善了实验数据与性能分析部分。通过系统性的对比实验,明确了在\red{SHA2-128f参数集}下,本方案实现了\red{62,239次/秒}的签名吞吐量,较Wang等人的方案提升了1.15倍。针对不同参数集的扩展实验表明,对于计算密集型的SHA2-128s参数集,吞吐量提升达到了1.33倍。

优化了学术表达方式与技术术语。减少了列表环境的使用,改为更连贯的段落叙述,提升了论文的学术性与可读性。精确定义了技术术语,如small(s)和fast(f)操作模式的明确解释,增强了论文的专业性。文中图表也进行了优化,增加了签名延迟分布的\blue{可视化分析},为读者提供更直观的性能对比。

\subsection{期刊投稿准备}

完成了面向\blue{IEEE Transactions on Circuits and Systems Part II: Express Briefs}的投稿准备工作:
撰写了专业的\blue{Cover Letter}。信中系统阐述了论文的三个主要贡献:\red{自适应线程分配框架}、\red{函数级并行架构}以及\red{全面的性能评估}。强调了本研究与期刊范围的高度契合性,特别是在密码硬件实现(DCS120B0)与密码架构(DCS120A5)两个领域。详细说明了研究成果的创新性与实用价值,突出了在量子安全密码学领域的前沿贡献。

按照期刊严格要求整理了投稿材料。IEEE TCAS-II对篇幅有严格限制——内容部分\red{4.5页},参考文献\red{0.5页},总计5页。针对此要求,对论文格式进行了专门调整,确保参考文献单独占据最后半页。创建了完整的提交清单,包括主文稿PDF、LaTeX源文件压缩包、利益冲突声明等所有必要文件。


% Replace standard bibliography commands with conditional version
\printbibliographyifcited[alpha]{../../paper}

\end{document}
