\documentclass{beamer}
\usepackage{../../typesetting/styles/slide-zh}
\usepackage{bookmark}

\title{\LARGE{周报}}
\subtitle{}
\author{}
\date{2025-05-12}

\begin{document}

% Title frame
\begin{frame}
  \titlepage
\end{frame}

% Outline frame
\begin{frame}{大纲}
  \tableofcontents
\end{frame}


\section{研究方向与进展}
\begin{frame}{第四篇论文研究方向}
  \begin{block}{研究聚焦}
    本周确定了第四篇论文的研究方向,聚焦于\blue{故障攻击防护技术研究}。通过对主要密码学会议和期刊的文献统计,发现72篇关于故障攻击的论文,其中\red{12篇专注于防护技术}。本论文将从以往的性能优化转向\red{安全实现},重点关注侧信道攻击与故障注入攻击的防护机制,尤其针对后量子密码算法。研究目标为开发高效、可验证且适用于资源受限环境的防护方法。
  \end{block}
   \begin{alertblock}{能有机合成硕士大论文不?}
    大论文\blue{软硬件优化实现}包含\blue{抗故障攻击的安全实现},可以和第二篇论文软件实现工作结合,实现更快更安全。
  \end{alertblock}
\end{frame}

\section{实验环境搭建}
\begin{frame}{故障注入实验环境}
  \textbf{环境构建.} 构建了基于电压毛刺的故障注入实验环境,包括目标设备(STM32F303芯片)、电压故障注入设备和示波器。实验流程为:目标设备发送触发信号,电压故障设备\red{注入电压故障},示波器捕获波形,主机对比密文以判定注入效果。

  % 添加实验装置图片
  \begin{figure}
    \centering
    \rotatebox{270}{\includegraphics[width=0.38\textwidth]{../../paper/FourthPaper/background/fig/fault_device.jpg}}
    \caption{故障注入实验装置。左侧为目标设备(STM32F303芯片),右上方为电压故障注入设备,右下方为捕获波形的示波器。}
  \end{figure}
\end{frame}

\begin{frame}{注入点分析}
  \begin{block}{注入点分析}
    通过捕获芯片电源与AES加密触发信号,确定第9轮加密的\blue{精确时间窗口(约4.405ms)},为故障注入提供了依据。
  \end{block}
  % 添加注入点分析波形图
  \begin{figure}
    \centering
    \includegraphics[width=0.5\textwidth]{../../paper/FourthPaper/background/fig/FaultPoint.png}
    \caption{故障注入点分析的捕获波形。A通道显示芯片工作电源,B通道显示AES加密触发信号。}
  \end{figure}
\end{frame}

\section{实验结果与分析}
\begin{frame}{实验结果}
  \begin{block}{初步测试}
    在AES加密过程中,尝试了超过50次电压故障注入,未观察到\red{错误的密文输出}。随后调整电压毛刺的注入时间$x$和持续时间$y$,在约3000次注入后,\red{成功观察到一条错误的密文输出}。而在\blue{公众号}参考参数下,注入3000次未见异常。阅读CHES论文\cite{TCHES:BozFocPal19},以提高注入成功率。
  \end{block}
  % 添加实验结果波形图
  \begin{figure}
    \centering
    \includegraphics[width=0.5\textwidth]{../../paper/FourthPaper/background/fig/AttackInject.png}
    \caption{AES加密过程中的故障注入攻击波形。蓝线急剧下降表示故障注入过程中的电压波动。}
  \end{figure}
\end{frame}

\begin{frame}{老师评语}
  \begin{alertblock}{故障攻击这些实验都是自己做出来的??}
    是自己做的,其中大部分是实验室设备,电压毛刺是淘宝买的。
  \end{alertblock}
 
  \begin{block}{下周计划}
    1)优化故障注入参数,提升\red{攻击成功率},并尝试通过差分故障分析提取密钥。
    2)跑通攻击流程后,阅读\blue{抗故障攻击的实现}文献。
  \end{block}
\end{frame}

\begin{frame}
  \frametitle{参考文献}
  \bibliographystyle{alpha}
  \bibliography{../../paper/FourthPaper/tex/abbrev3,../../paper/FourthPaper/tex/crypto_custom}
\end{frame}

\end{document}
