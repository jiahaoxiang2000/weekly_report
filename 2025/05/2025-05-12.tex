\documentclass{article}
\usepackage{../../typesetting/styles/report-zh}
\usepackage{threeparttable} % Add this package for tablenotes environment

\setCJKmainfont{Noto Serif CJK SC} % Main Chinese font (Songti)
\setCJKsansfont{Noto Sans CJK SC} % Sans-serif Chinese font
\setCJKmonofont{Noto Sans Mono CJK SC} % Monospaced Chinese font (uncomment if needed)

% Set document information
\title{周报~向嘉豪 (\today)}
\author{向嘉豪}
\date{\today}

\begin{document}

\maketitle

\begin{abstract}
  本周主要确定了第四篇论文的研究方向,\blue{聚焦于故障攻击防护技术研究}。构建了基于\blue{电压毛刺}的故障注入实验环境,并进行了初步测试。在调整注入参数后,\red{成功观察到一个错误的密文输出}。下周将继续\blue{优化故障注入参数},提高\red{攻击成功率},并尝试通过\blue{差分故障分析}提取密钥。
\end{abstract}

\begin{weekplan}
1) 提高故障注入成功率 2)差分故障分析出密钥
\end{weekplan}

\section{第四篇论文}


基于对主要密码学会议和期刊(ASIACRYPT 1990-2024、CRYPTO 1981-2024、EUROCRYPT 1982-2024、TCHES 2018-2024、ToSC 2016-2024)的文献统计分析,我们发现了\blue{72篇关于故障攻击(Fault Attack)的论文},其中有\red{12篇(约16.7\%)专注于故障攻击防护技术}。在前三篇论文中,我们主要关注密码学算法的\blue{性能优化},而在第四篇论文中,我们将转向\red{安全实现},尤其是针对\blue{侧信道攻击}和\red{故障注入攻击}的防护机制。计划深入研究针对现代密码系统(特别是\blue{后量子密码算法})的故障攻击防护技术。\blue{研究将聚焦于开发可验证、高效且适用于资源受限环境的故障攻击防护方法}。为此需要搭建一个\red{故障注入实验环境}。

\subsection{实验环境}

故障注入攻击的工作流程可概括为:目标设备发送\blue{触发信号}至电压故障设备,电压故障设备向目标设备\red{注入电压故障},同时示波器\blue{捕获波形},主机接收密文并与预期密文进行比较,从而确定故障注入是否成功。启发于老师转发的\blue{PADNA2025},我们建立了完整的故障注入实验环境(基于\red{电压毛刺})。\blue{该环境由三个主要组件构成}:目标设备(\blue{STM32F303芯片})、电压故障注入设备和用于捕获波形的示波器。实验设置如下:

\begin{figure}[htbp]
    \centering
    \includegraphics[width=0.34\textwidth]{../../paper/FourthPaper/background/fig/fault_device.jpg}
    \caption{故障注入实验装置。左侧为目标设备(STM32F303芯片),右上方为电压故障注入设备,右下方为捕获波形的示波器。}
    \label{fig:fault_device}
\end{figure}

首先,我们识别了\blue{潜在的故障注入点}。通过捕获芯片工作电源的A通道和AES加密触发信号的B通道,我们观察到触发信号后第9轮加密的运行时间接近\blue{4.405ms}。这为故障注入提供了\red{精确的时间窗口},如图\ref{fig:fault_point}所示。

\begin{figure}[htbp]
    \centering
    \includegraphics[width=0.5\textwidth]{../../paper/FourthPaper/background/fig/FaultPoint.png}
    \caption{故障注入点分析的捕获波形。A通道显示芯片工作电源,B通道显示AES加密触发信号。}
    \label{fig:fault_point}
\end{figure}

在攻击过程中,我们采用了\red{电压故障攻击}方法。我们在AES加密过程中尝试了超过\blue{50次电压故障注入},特别针对接近第9轮的时间段。如图\ref{fig:attack_inject}所示,\blue{蓝线急剧下降}表示故障注入过程中的电压波动。尽管进行了多次尝试,我们未能观察到任何\red{错误的密文输出},表明故障注入\red{未能成功}。

\begin{figure}[htbp]
    \centering
    \includegraphics[width=0.5\textwidth]{../../paper/FourthPaper/background/fig/AttackInject.png}
    \caption{AES加密过程中的故障注入攻击波形。蓝线急剧下降表示故障注入过程中的电压波动。}
    \label{fig:attack_inject}
\end{figure}

为此我们对\blue{电压毛刺的注入时间}$x$\blue{和持续时间}$y$\blue{进行调整},在大约\blue{3000次注入}后,我们观察到了\red{一条错误的密文输出},而在PADNA2025公众号上提供的$(x,y)$参数下,我们注入3000次(约1个小时)未观察到任何错误的密文输出。为\blue{提高注入的成功率},我们正在阅读\cite{TCHES:BozFocPal19}。



% Replace standard bibliography commands with conditional version
\printbibliographyifcited[alpha]{../../paper/FourthPaper/tex/abbrev3,../../paper/FourthPaper/tex/crypto_custom}

\end{document}
