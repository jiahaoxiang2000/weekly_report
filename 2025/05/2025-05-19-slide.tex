\documentclass{beamer}
\usepackage{../../typesetting/styles/slide-zh}
\usepackage{bookmark}

\title{\LARGE{周报}}
\subtitle{}
\author{}
\date{\today}

\begin{document}

% Title frame
\begin{frame}
  \titlepage
\end{frame}

% Outline frame
\begin{frame}{大纲}
  \tableofcontents
\end{frame}


\section{第三篇论文投稿情况}
\begin{frame}{投稿结果与反馈}
  \begin{block}{TCAS-II投稿结果}
    本周(5月14日)收到TCAS-II期刊对第三篇论文的\red{拒稿决定}。该论文于5月8日提交,经过6天审核即被拒绝。编辑给出的明确拒稿理由为:\textit{“Unfortunately, this paper does not fit well within the scope of the journal. The scientific content related to the circuits \& system field is very limited. As it is, it seems more suitable for the Computer community.”}(论文与期刊范围不符,电路与系统领域相关的科学内容有限,更适合计算机领域社区)。
  \end{block}
\end{frame}

\section{投稿策略分析}
\begin{frame}{投稿决策回顾与调研}
  \begin{block}{决策回顾}
    最初选择TCAS-II,因其2022年曾发表相关主题论文。然而,编辑反馈表明\red{期刊范围定位可能已调整},或对电路系统内容要求提高。对于\blue{增加电路内容并重新投递}的建议,因缺乏电路设计背景及时间成本,评估后认为\red{风险较高},决定采纳编辑建议,转投更适合的计算机领域期刊。
  \end{block}
  \begin{block}{期刊调研}
    通过系统性调研,拓展候选范围至IEEE、\blue{Springer}及\blue{ACM}出版社。适合短篇论文(letter type)的期刊多为\red{Q3至Q4}级别。综合评估后,确定三个主要候选:IEEE Computer Architecture Letters(\blue{契合度最高})、IEEE Communications Letters(\red{影响因子较高},但范围偏信息编码)、IEEE Embedded Systems Letters(与嵌入式系统安全相关)。
  \end{block}
\end{frame}

\begin{frame}{投稿计划与内容调整}
  \begin{block}{投稿计划}
    后续将分析上述期刊的\blue{最新出版内容},重点关注密码学算法实现与安全相关文献。为避免再次因范围不符被拒,计划\red{先行邮件咨询编辑},确认论文是否符合期刊范围。此举有助于节省投稿时间并提升接受概率。
  \end{block}
  \begin{block}{内容调整}
    当前研究以短篇论文(letter type)形式完成,若扩展为常规论文将显著延长周期。考虑到\red{时间效率},优先考虑letter类型期刊。后续将根据目标期刊要求,调整\red{引言和背景},突出研究与期刊侧重点的契合性,并重新评估\blue{关键词选择},以便更准确反映研究内容与领域定位。
  \end{block}
\end{frame}


\begin{frame}{老师评语}
\begin{alertblock}{可以投,先试一下,要尽快保证在9月份有篇好论文接受}
    对文章小修,尽快投出
\end{alertblock}

  \begin{block}{主要任务}
    1)修改论文 2)投递期刊 
  \end{block}
\end{frame}

% \begin{frame}
%   \frametitle{参考文献}
%   \bibliographystyle{alpha}
%   \bibliography{../../paper/FourthPaper/tex/abbrev3,../../paper/FourthPaper/tex/crypto_custom}
% \end{frame}

\end{document}
