\documentclass{article}
\usepackage{../../typesetting/styles/report-zh}
\usepackage{threeparttable} % Add this package for tablenotes environment

\setCJKmainfont{Noto Serif CJK SC} % Main Chinese font (Songti)
\setCJKsansfont{Noto Sans CJK SC} % Sans-serif Chinese font
\setCJKmonofont{Noto Sans Mono CJK SC} % Monospaced Chinese font (uncomment if needed)

% Set document information
\title{周报~向嘉豪 (\today)}
\author{向嘉豪}
\date{\today}

\begin{document}

\maketitle

\begin{abstract}
  本周接获TCAS-II期刊对第三篇论文的\red{拒稿通知},编辑明确表示论文与期刊范围不符,建议投递至计算机领域期刊。针对此反馈,进行了\blue{全面的期刊调研},着重分析适合短篇论文(letter type)的出版选项。通过对\blue{IEEE、Springer及ACM}出版社相关期刊的系统评估,确定了三个主要候选期刊。同时对投稿策略进行了\red{全面重新评估},综合考量了审稿周期、接受概率与论文结构调整等因素。
\end{abstract}

\begin{weekplan}
确定第三篇论文的最终目标期刊;依据目标期刊要求调整论文内容和格式;准备投稿前的编辑咨询通信。
\end{weekplan}

\section{第三篇论文投稿情况}

本周(5月14日)收到TCAS-II期刊对第三篇论文的\red{拒稿决定}。该论文于5月8日提交,经过6天审核即被拒绝。编辑给出的明确拒稿理由为:``Unfortunately, this paper does not fit well within the scope of the journal. The scientific content related to the circuits \& system field is very limited. As it is, it seems more suitable for the Computer community."(论文与期刊范围不符,电路与系统领域相关的科学内容有限,更适合计算机领域社区)。

\section{论文重新提交策略分析}

\subsection{初始投稿决策回顾}

通过深入分析,我们对最初选择TCAS-II的决策进行了反思。该选择主要基于该期刊曾于2022年发表过类似主题的研究:《Efficient implementation of AES-CTR and AES-ECB on GPUs with applications for high-speed FrodoKEM and exhaustive key search》。然而,编辑的反馈明确指出,\red{当前期刊的范围定位可能已经调整},或者对电路系统内容的要求提高。

对于学长提出的\blue{增加电路内容并重新投递TCAS-II}的建议,经过审慎评估认为该方案面临重大挑战:首先,缺乏电路设计专业背景,难以在短期内补充高质量的电路系统内容;其次,参考的2022年发表论文亦不含显著电路内容但获接受,表明\blue{期刊的评审标准可能正在收紧或转变}。考虑到重新修改后\red{仍可能被拒}的风险及时间成本,决定采纳编辑建议,将论文转投更适合的计算机领域期刊。

\subsection{期刊调研与选择}

通过系统性的期刊调研,我们拓展了候选范围,涵盖IEEE、\blue{Springer}及\blue{ACM}出版社。研究表明,适合短篇论文(letter type)的专业期刊多数位于\red{Q3至Q4}级别。经综合评估期刊范围匹配度、审稿周期与影响力,确定了三个主要候选期刊:IEEE Computer Architecture Letters(与研究内容\blue{契合度最高},曾发表多篇硬件加速密码学算法研究)、IEEE Communications Letters(\red{影响因子相对较高},但范围适配性存在风险)及IEEE Embedded Systems Letters(与嵌入式系统安全有关联)。

后续将深入分析这些期刊的\blue{最新出版内容},特别关注密码学算法实现与安全相关文献,以进一步评估适配性。为避免再次因范围不匹配而被拒,计划\red{首先通过正式邮件咨询各期刊编辑},明确确认论文是否符合期刊范围要求。此预先咨询策略可有效节省投稿流程时间并提高接受概率。

\subsection{投稿形式考量}

在确定投稿策略时,需权衡\blue{论文形式}与\red{时间效率}。当前研究以短篇论文(letter type)形式完成,若转为常规长度论文需大幅扩展内容,将显著延长准备周期。考虑到短篇论文通常具有约三个月的接受周期,而常规论文可能需要六个月或更长时间,基于当前研究进度和时间规划,决定维持短篇论文形式,优先考虑接受letter类型的期刊。

后续将根据所选期刊的具体要求,适当调整论文的\red{引言和背景部分},更明确地突出研究与目标期刊侧重点的契合性,同时保持研究核心内容的完整性与科学严谨性。同时,计划重新评估\blue{关键词选择},确保能更准确地反映研究内容与领域定位,有助于期刊编辑匹配合适的审稿人。


% Replace standard bibliography commands with conditional version
\printbibliographyifcited[alpha]{../../paper/FourthPaper/tex/abbrev3,../../paper/FourthPaper/tex/crypto_custom}

\end{document}
