\documentclass{beamer}
\usepackage{../../typesetting/styles/slide-zh}
\usepackage{bookmark}

\title{\LARGE{周报}}
\subtitle{}
\author{}
\date{\today}


\begin{document}

% Title frame
\begin{frame}
  \titlepage
\end{frame}

% Outline frame
\begin{frame}{大纲}
  \tableofcontents
\end{frame}

\section{相关工作技术调研}
\begin{frame}{功耗与故障攻击的联合防护}
  \begin{block}{联合攻击防护的关键问题}
    Saha等人~\cite{COSADE:SahBagJapMuk21}在COSADE 2021上发表的"Divided We Stand, United We Fall"深入分析了SCA+SIFA反措施的安全性问题,\red{揭示了单独设计的防护措施在联合攻击下可能失效的关键问题}。
  \end{block}
  
  \begin{block}{统一防护机制}
    Miškovský和Kubátová~\cite{IEEE-TVLSI:MisKub21}提出了面积高效的掩码与故障容错架构,通过减少冗余度实现安全性与硬件开销的平衡。Belenky等人~\cite{TCHES:BelBugAzr22}的RAMBAM方案将乘法掩码与冗余机制结合,增强AES实现的故障抗性。
  \end{block}

  \begin{block}{集成对策方案}
    Ramezanpour等人~\cite{IEEE-TCAD:RamAmp20}提出的RS-MASK方案作为针对功耗分析和故障分析的集成对策,\red{使用随机空间掩码技术同时抵御两类攻击}。
  \end{block}
\end{frame}

\begin{frame}{后量子密码学的故障攻击防护}
  \begin{block}{后量子时代的新挑战}
    Howe等人~\cite{IEEE-HOST:HowKhaMarNor19}在IEEE TC上提出了针对格基密码学中误差采样器的故障攻击对策。这项工作专门解决了后量子密码构造中的独特漏洞,\blue{为后量子时代的安全芯片设计提供了重要的理论基础}。
  \end{block}
  
  \begin{block}{格基密码的脆弱性}
    后量子算法的独特结构引入了新的攻击面。特别是在误差采样和格运算过程中,\red{故障可能导致格结构的破坏,从而暴露私钥信息},要求重新设计针对后量子密码的故障防护机制。
  \end{block}
\end{frame}

\section{当前技术挑战与研究缺口}
\begin{frame}{故障注入参数优化的复杂性}
  \begin{block}{参数空间爆炸问题}
    Krček和Ordas~\cite{ESORICS:KrcOrd24}的研究表明,\red{激光故障注入的参数空间极其庞大},传统的穷举搜索方法效率低下。他们提出了基于遗传算法的多样性优化策略,但仍然面临收敛速度和全局最优解的挑战。
  \end{block}
  
  \begin{block}{理论与实践的差距}
    Toprakhisar等人~\cite{ESORICS:TopNikNik24}在ESORICS 2024上系统梳理了故障对手模型的参数化问题,强调了理论模型与实际攻击能力之间的差距。
  \end{block}
\end{frame}

\begin{frame}{形式化验证的扩展性问题}
  \begin{block}{状态空间爆炸挑战}
    Tollec等人~\cite{TCHES:THNABC24}在TCHES上建立了k-故障抗性分区的理论基础,但在复杂系统中的扩展性仍然有限。\blue{当系统规模增大时,状态空间爆炸问题变得严重},需要开发更加高效的符号执行和模型检验技术。
  \end{block}
  
  \begin{block}{实用化瓶颈}
    对于现代处理器中包含的数百万门电路,现有方法的计算复杂度呈指数级增长,\red{这是限制k-故障抗性分区实用化的核心瓶颈}。
  \end{block}
\end{frame}

\begin{frame}{多重攻击向量的统一建模挑战}
  \begin{block}{联合攻击的威胁}
    现有研究往往独立考虑功耗攻击和故障攻击的防护,对于两类攻击联合实施时的安全性分析相对薄弱。Saha等人~\cite{COSADE:SahBagJapMuk21}的工作深刻揭示了这一问题:\red{许多单独设计的SCA+SIFA防护措施在面对联合攻击时会失效}。
  \end{block}
\end{frame}


\begin{frame}{老师评语}
    \begin{alertblock}{再看看最新顶刊论文,做后量子是可以的}
      往\red{后量子算法}\blue{实现}方向深入调研
    \end{alertblock}
\end{frame}

\begin{frame}[allowframebreaks]
  \frametitle{参考文献}
  \bibliographystyle{alpha}
  \bibliography{../../paper/FourthPaper/tex/abbrev3,../../paper/FourthPaper/tex/crypto_custom,../../paper/FourthPaper/tex/biblio}
\end{frame}

\end{document}
