\documentclass{article}
\usepackage{../../typesetting/styles/report-zh}
\usepackage{threeparttable} % Add this package for tablenotes environment

\setCJKmainfont{Noto Serif CJK SC} % Main Chinese font (Songti)
\setCJKsansfont{Noto Sans CJK SC} % Sans-serif Chinese font
\setCJKmonofont{Noto Sans Mono CJK SC} % Monospaced Chinese font (uncomment if needed)

% Set document information
\title{周报~向嘉豪 (\today)}
\author{向嘉豪}
\date{\today}

\begin{document}

\maketitle

\begin{abstract}
本周报告分析了后量子密码学(PQC)算法标准化现状,重点关注数字签名算法(DSA)在网络协议迁移中面临的技术挑战。通过对NIST标准化进程、协议迁移测试结果以及相关研究工作的系统性分析,确定了ML-DSA与MQTT协议结合作为物联网环境下后量子签名算法研究的主要方向。其中,\red{后量子签名算法的大签名尺寸是协议迁移的主要瓶颈},而物联网协议的资源约束环境进一步放大算法优化实现的研究意义。
\end{abstract}

\begin{weekplan}
1) 制定ML-DSA在MQTT协议中的实现优化研究计划
\end{weekplan}

\section{研究进展分析}

\subsection{后量子密码学标准化现状}

NIST后量子密码学标准化进程已基本完成,形成了两大类核心算法标准。在密钥封装机制(KEM)方面,\blue{FIPS 203(ML-KEM)基于CRYSTALS-Kyber已于2024年正式发布},为密钥交换提供了标准化解决方案。数字签名算法(DSA)领域则包含三个标准:FIPS 204(ML-DSA)基于CRYSTALS-Dilithium、FIPS 205(SLH-DSA)基于SPHINCS+,以及正在开发中的FIPS 206(FN-DSA)基于FALCON,预计2025年夏季发布。

2025年3月的重要进展包括HQC算法被选中进行标准化以及14个第二轮On-Ramp签名候选算法的确定。这些发展表明后量子密码学正从理论研究转向实际部署阶段,但同时也暴露出算法多样性带来的选择复杂性。
重要的是,\red{旧签名和密钥封装算法标准将于2035年废止,迫切需要向后量子安全算法迁移}。

\subsection{协议迁移技术挑战分析}

基于NCCoE SP 1800-38C的协议迁移测试结果,\red{数字签名算法(DSA)迁移比密钥交换迁移面临更严峻的挑战}。关键发现包括:

在TLS 1.3协议中,Kyber密钥交换显示出优异的性能表现,Kyber-768实现了681次握手/秒的吞吐量,与经典P384算法的223次握手/秒相比具有明显优势。然而,后量子签名算法的大尺寸特性带来了显著的网络开销。\blue{Dilithium证书大小达到18-22 KB,在QUIC协议中触发了额外的往返传输},导致放大保护机制启动和拥塞控制窗口限制。

SSH协议由于其多轮传输设计,对后量子签名的性能影响相对较小。但在认证支持方面,当时测试中仅有OQS-OpenSSH提供了后量子认证功能,显示出实现生态系统的不成熟。

\subsection{DSA算法与协议适配性评估}

通过对三种标准化DSA算法的深入分析,发现各算法在不同应用场景下的适用性存在显著差异:

ML-DSA(FIPS 204)作为NIST的主要推荐算法,在性能和安全性之间提供了良好的平衡。\blue{GPU加速研究显示cuML-DSA在服务器GPU上实现了170.7×到294.2×的性能提升},通过深度优先稀疏三元多项式乘法优化和分支消除方法,为高吞吐量服务器环境提供了可行的解决方案。

FN-DSA(FIPS 206)虽然提供最小的签名尺寸和快速验证,但面临严重的安全挑战。\red{2025年发现的Rowhammer攻击显示单个比特翻转可以通过数亿次签名恢复完整密钥},浮点运算敏感性和侧信道漏洞进一步限制了其实际部署的安全性。

SLH-DSA(FIPS 205)基于哈希函数的保守安全基础,但其\red{极大的签名尺寸和较慢的签名速度限制了实际应用场景}。硬件加速研究显示SLotH实现可达到300×的性能提升,但仍难以满足高频次签名需求和资源受限环境。

\subsection{ML-DSA与MQTT协议研究方向确定}

经过综合评估,确定ML-DSA与MQTT协议结合作为主要研究方向,基于以下关键考虑:

\textbf{算法选择理由:}ML-DSA作为NIST主要推荐的后量子签名算法,具有相对成熟的安全分析和实现基础。其在GPU加速方面的研究进展为服务器端优化提供了参考,同时其模格假设相比其他算法具有更好的理论基础。

\textbf{协议选择理由:}物联网环境下的MQTT协议迁移研究相对不足,\red{现有研究主要集中在KEM-MQTT实现上,直接的ML-DSA与IoT协议集成研究几乎空白}。2025年的KEM-MQTT研究在8位AVR设备上优化实现,并发表于最新25-CCS(CCF-A)安全顶会,证明了在资源受限环境下实现后量子安全的研究价值。

\textbf{研究机会识别:}物联网设备的资源约束特性为算法优化提供了独特的研究机会。现有的嵌入式系统研究(如pqm4基准测试框架)为ML-DSA在ARM Cortex-M4上的性能评估提供了基础,但缺乏针对IoT协议的专门优化。侧信道安全研究显示了在微控制器上实现安全ML-DSA的挑战和对策,为安全实现提供了指导。

\textbf{实际应用价值:}随着IoT设备数量的快速增长和量子计算威胁的逐步现实化,\blue{为资源受限的IoT环境提供量子安全的认证机制具有重要的实际意义}。传统的证书链方法在IoT环境下面临带宽和存储限制,需要创新的轻量级认证策略,也是未来研究的方向之一。

% Replace standard bibliography commands with conditional version
\printbibliographyifcited[alpha]{../../paper/FourthPaper/tex/abbrev3,../../paper/FourthPaper/tex/biblio}

\end{document}
